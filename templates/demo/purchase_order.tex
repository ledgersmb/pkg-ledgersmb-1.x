<?lsmb FILTER latex { format="$FORMAT(pdflatex)" } -?>
\documentclass{scrartcl}
\usepackage[utf8]{inputenc}
\usepackage{tabularx}
\usepackage{longtable}
\setlength\LTleft{0pt}
\setlength\LTright{0pt}
\usepackage[letterpaper,top=2cm,bottom=1.5cm,left=1.1cm,right=1.5cm]{geometry}
\usepackage{graphicx}

\begin{document}


\pagestyle{myheadings}
\thispagestyle{empty}

\fontfamily{cmss}\fontsize{10pt}{12pt}\selectfont

<?lsmb INCLUDE letterhead ?>


\markboth{<?lsmb company ?>\hfill <?lsmb ordnumber ?>}{<?lsmb company ?>\hfill <?lsmb ordnumber ?>}

\vspace*{0.5cm}

\parbox[t]{.5\textwidth}{
\textbf{To}
\vspace{0.3cm}

<?lsmb name ?>

<?lsmb address1 ?>

<?lsmb address2 ?>

<?lsmb city ?>
<?lsmb IF state ?>
\hspace{-0.1cm}, <?lsmb state ?>
<?lsmb END ?>
<?lsmb zipcode ?>

<?lsmb country ?>

\vspace{0.3cm}

<?lsmb IF contact ?>
<?lsmb text('Attn: [_1]', contact) ?>
\vspace{0.2cm}
<?lsmb END ?>

<?lsmb IF vendorphone ?>
<?lsmb text('Tel: [_1]', vendorphone) ?>
<?lsmb END ?>

<?lsmb IF vendorfax ?>
<?lsmb text('Fax: [_1]', vendorfax) ?>
<?lsmb END ?>

<?lsmb email ?>
}
\parbox[t]{.5\textwidth}{
\textbf{Ship To}
\vspace{0.3cm}

<?lsmb shiptoname ?>

<?lsmb shiptoaddress1 ?>

<?lsmb shiptoaddress2 ?>

<?lsmb shiptocity ?>
<?lsmb IF shiptostate ?>
\hspace{-0.1cm}, <?lsmb shiptostate ?>
<?lsmb END ?>
<?lsmb shiptozipcode ?>

<?lsmb shiptocountry ?>

\vspace{0.3cm}

<?lsmb IF shiptocontact ?>
<?lsmb text('Attn: [_1]', shiptocontact) ?>
\vspace{0.2cm}
<?lsmb END ?>

<?lsmb IF shiptophone ?>
<?lsmb text('Tel: [_1]', shiptophone) ?>
<?lsmb END ?>/

<?lsmb IF shiptofax ?>
<?lsmb text('Fax: [_1]', shiptofax) ?>
<?lsmb END ?>

<?lsmb shiptoemail ?>
}
\hfill

\vspace{1cm}

\textbf{\MakeUppercase{<?lsmb text('Purchase Order') ?>}}
\hfill

\vspace{1cm}
\begin{tabularx}{\textwidth}{*{6}{|X}|} \hline
  \textbf{<?lsmb text('Order #') ?>} & \textbf{<?lsmb text('Date') ?>}
   & \textbf{<?lsmb text('Required by') ?>} & \textbf{<?lsmb text('Contact') ?>}
   & \textbf{<?lsmb text('Shipping Point') ?>}
   & \textbf{<?lsmb text('Ship via') ?>} \\ [0.5ex]
  \hline
  <?lsmb ordnumber ?> & <?lsmb orddate ?> & <?lsmb reqdate ?> & <?lsmb employee ?> & <?lsmb shippingpoint ?> & <?lsmb shipvia ?> \\
  \hline
\end{tabularx}

\vspace{1cm}

\begin{longtable}{@{\extracolsep{\fill}}llrlrr@{\extracolsep{0pt}}}
  \textbf{<?lsmb text('Number') ?>} & \textbf{<?lsmb text('Description') ?>}
  & \textbf{<?lsmb text('Qty') ?>} &
    \textbf{<?lsmb text('Unit') ?>} & \textbf{<?lsmb text('Price') ?>}
   & \textbf{<?lsmb text('Amount') ?>} \\
\endhead
<?lsmb FOREACH number ?>
<?lsmb lc = loop.count - 1 ?>
  <?lsmb number.${lc} ?> &
  <?lsmb item_description.${lc} ?> &
  <?lsmb qty.${lc} ?> &
  <?lsmb unit.${lc} ?> &
  <?lsmb sellprice.${lc} ?> &
  <?lsmb linetotal.${lc} ?> \\
<?lsmb END ?>
\end{longtable}


\parbox{\textwidth}{
\rule{\textwidth}{2pt}

\vspace{0.2cm}

\hfill
\begin{tabularx}{7cm}{Xr@{\hspace{1cm}}r@{}}
  & Subtotal & <?lsmb subtotal ?> \\
<?lsmb FOREACH tax ?>
<?lsmb lc = loop.count - 1 ?>
  & <?lsmb taxdescription.${lc} ?> on <?lsmb taxbase.${lc} ?> & <?lsmb tax.${lc} ?>\\
<?lsmb END ?>
  \hline
  & Total & <?lsmb ordtotal ?>\\
\end{tabularx}

\vspace{0.3cm}

\hfill
  <?lsmb text('All prices in [_1].', currency) ?>

\vspace{12pt}

<?lsmb FOREACH P IN notes.split('\n{2,}') ?>
<?lsmb P ?>\medskip

<?lsmb END ?>

}


%\renewcommand{\thefootnote}{\fnsymbol{footnote}}

%\footnotetext[1]{\tiny }

\end{document}
<?lsmb END ?>
